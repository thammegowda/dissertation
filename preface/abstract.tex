%Most naturally occurring distributions are imbalanced; while some categories are very frequent, others are very rare.


Machine translation (MT) is one of the earliest and most successful applications of natural language processing. Many MT services have been deployed via web and smartphone apps, enabling communication and information access across the globe by bypassing language barriers. However, MT is not yet a solved problem. MT services that cover the most languages cover only about a hundred; thousands more are currently unsupported. Even for the currently supported languages, the translation quality is far from perfect.

A key obstacle in our way to achieving usable MT models for any language is data imbalance. On the one hand, machine learning techniques perform subpar on rare categories, having only a few to no training examples. On the other hand, natural language datasets are inevitably imbalanced with a long tail of rare types. The rare types carry more information content, and hence correctly translating them is crucial. In addition to the rare word types, rare phenomena also manifest in other forms as rare languages and rare linguistic styles.

Our contributions towards advancing rare phenomena learning in MT are four-fold: (1) We show that MT models have much in common with classification models, especially regarding the data imbalance and frequency-based biases. We describe a way to reduce the imbalance severity during the model training. (2) We show that the currently used automatic evaluation metrics overlook the importance of rare words. We describe an interpretable evaluation metric that treats important words as important. (3) We propose methods to evaluate and improve translation robustness to rare linguistic styles such as partial translations and language alternations in inputs. (4) Lastly, we present a set of tools intended to advance MT research across a wider range of languages. Using these tools, we demonstrate 600 languages to English translation, thus supporting 500 more rare languages currently unsupported by others.